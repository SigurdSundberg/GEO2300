\documentclass[10pt, a4paper]{amsart}
% \documentclass[10pt,showpacs,preprintnumbers,footinbib,amsmath,amssymb,aps,prl,twocolumn,groupedaddress,superscriptaddress,showkeys]{revtex4-1}
\usepackage[]{graphicx}
\usepackage[]{hyperref}
\usepackage[]{physics}
\usepackage[]{listings}
\usepackage[T1]{fontenc}
\usepackage{color}
\usepackage[ruled,vlined]{algorithm2e}

\usepackage{amsmath, amsfonts, amssymb}
\usepackage{listings}

% Definition of code and how the code should look.
\usepackage{xcolor}
\definecolor{codegreen}{rgb}{0,0.6,0}
\definecolor{codegray}{rgb}{0.5,0.5,0.5}
\definecolor{codepurple}{rgb}{0.58,0,0.82}
\definecolor{backcolour}{rgb}{0.95,0.95,0.92}
\definecolor{mygreen}{rgb}{0,0.6,0}
\definecolor{mymauve}{rgb}{0.58,0,0.82}


\lstdefinestyle{python3}{
    backgroundcolor=\color{backcolour},
    commentstyle=\color{codegreen},
    keywordstyle=\color{magenta},
    numberstyle=\tiny\color{codegray},
    stringstyle=\color{codepurple},
    basicstyle=\ttfamily\footnotesize,
    breakatwhitespace=false,
    breaklines=true,
    captionpos=b,
    keepspaces=true,
    numbers=left,
    numbersep=5pt,
    showspaces=false,
    showstringspaces=false,
    showtabs=false,
    tabsize=2
}
% End def.



\lstset{ %
  backgroundcolor=\color{white},   % choose the background color; you must add \usepackage{color} or \usepackage{xcolor}
  basicstyle=\footnotesize,        % the size of the fonts that are used for the code
  breakatwhitespace=false,         % sets if automatic breaks should only happen at whitespace
  breaklines=true,                 % sets automatic line breaking
  captionpos=b,                    % sets the caption-position to bottom
  commentstyle=\color{mygreen},    % comment style
  deletekeywords={...},            % if you want to delete keywords from the given language
  escapeinside={\%*}{*)},          % if you want to add LaTeX within your code
  extendedchars=true,              % lets you use non-ASCII characters; for 8-bits encodings only, does not work with UTF-8
  frame=single,	                   % adds a frame around the code
  keepspaces=true,                 % keeps spaces in text, useful for keeping indentation of code (possibly needs columns=flexible)
  keywordstyle=\color{blue},       % keyword style
  language=python,                    % the language of the code
  style = python3,
  otherkeywords={*,...},           % if you want to add more keywords to the set
  rulecolor=\color{black},         % if not set, the frame-color may be changed on line-breaks within not-black text (e.g. comments (green here))
  showspaces=false,                % show spaces everywhere adding particular underscores; it overrides 'showstringspaces'
  showstringspaces=false,          % underline spaces within strings only
  showtabs=false,                  % show tabs within strings adding particular underscores
  stepnumber=2,                    % the step between two line-numbers. If it's 1, each line will be numbered
  stringstyle=\color{mymauve},     % string literal style
  tabsize=2,	                     % sets default tabsize to 2 spaces
}

\title[Problem set 1]{Problem set 1: GEO2300: \\
\normalsize{Due: 16 Sept. 2020} \\
  \hrulefill\small{ GEO2300: Fysiske prosesser i geofag }\hrulefill}

\author[Sundberg]{Sigurd Sandvoll Sundberg \\
  \href{https://https://github.com/SigurdSundberg/GEO2300/}{\texttt{github.com/sigurdsundberg}}}

\begin{document}

\begin{titlepage}
\maketitle
\tableofcontents
\end{titlepage}

\section{Problem 1: Matricies}
In this section we will study the properties of matricies. How they are used to solve a system of linear equations, inverting matricies and finding eiegenvalues and eigenvectors, and lastly expressing matricies as a sum of eigenvectors and exponentials with corresponding eigenvalues. 

First we will look at solving the following system of linear equations as a matrix equation
\begin{equation}\label{eq:sys_linear}
\begin{split}
	x + 2y + z &= -1\\
	2x - y + 3z &= -5 \\
	-x + 3y- z &= 6
\end{split}
\end{equation}
We can rewrite Eq: \ref{eq:sys_linear} on the form $\mathbf{A}\vec{x} = \vec{b}$
\begin{equation}
	\begin{bmatrix}
		1 & 2 & 1 & -1\\
		2 & -1 & 3 & -5 \\
		-1 & 3 & -1 & 6
	\end{bmatrix}
	\begin{bmatrix}
	x\\y\\z
	\end{bmatrix}
	=
	\begin{bmatrix}
	1\\-5\\6
	\end{bmatrix}
\end{equation}
We can rewrite this system on the form $[\mathbf{A} : \vec{b}]$ and solve the augmented matrix by finding the row reduced echelon form. We then have 
\begin{align}
 \left[
 \begin{array}{ccc|c}
 1 & 2 & 1 & 1 \\
 2 & -1 & 3 & -5\\
-1 & 3 & -1 & 6
 \end{array}
 \right] &\sim
 \left[
 \begin{array}{ccc|c}
 1 & 2 & 1 & 1 \\
 0 & -5 & 1 & -7\\
0 & 5 & 0 & 7
 \end{array}
 \right]\\
 \left[
 \begin{array}{ccc|c}
 1 & 2 & 1 & 1 \\
 0 & 1 & 0 & 7/5\\
0 & 0 & 1 & 0
 \end{array}
 \right] &\sim 
 \left[
 \begin{array}{ccc|c}
 1 & 0 & 0 & -9/5 \\
 0 & 1 & 0 & 7/5\\
0 & 0 & 1 & 0
 \end{array}
 \right]
 \end{align}
 giving us that the solution to the set of linear equations given in Eq: \ref{eq:sys_linear} is given by 
 \begin{align*}
  x &= -9/5\\
  y &= 7/5\\
  z &= 0
 \end{align*}
 This can easily be confirmed numerically by doing $rref([A : b])$. 
 
 Second, we will look at inversing matrixes, both analytically for the 2x2-matrix and numerically for the others. Looking at the 2x2-matrix we have that in the general case 
 \begin{equation*}
 	A = 
 	\begin{bmatrix}
 		a & b\\
 		c & d	
 	\end{bmatrix}, \quad A^{-1} = \frac{1}{ad-bc}
 	\begin{bmatrix}
 	d & -b\\
 	-c & a
 	\end{bmatrix}
 \end{equation*}
 We then have for our 2x2-matrix as follows
 \begin{equation}
 A = 
 \begin{bmatrix}
 1 & 2 \\
 -1 & 3
 \end{bmatrix}, \quad A^{-1} = \frac{1}{3\cdot 1 - (-1\cdot 2)}
 \begin{bmatrix}
 3 & -2 \\
 1 & 1
 \end{bmatrix}
 = \frac{1}{5}
 \begin{bmatrix}
 	3 & -2 \\
 	1 & 1
 \end{bmatrix}
 \end{equation}
 For the two next matricies being a 3x3 and 4x4 matrix, which are fairly tedious to solve by hand, we will use functions in Python 3 to do the inverting for us, spesifically the Numpy library. 
 
\lstinputlisting[firstline=2, lastline=8]{"./code/1b_inverse.py"}

Which gives the following output
 
\section{Problem 2: Poisueille Flow}
\section{Problem 3: More finite differences}


%%% footnote
% rainforest\footnote{Writing out a general case will also take up more paper
% space}.

%%% Matrix with line through between last elements
% \begin{equation}
% \left[
% \begin{array}{cccc|c}
% 1 & c_1/\beta_1 & 0 & 0 & \tilde{f}_1 \\
% 0 & 1 & c_2/\beta_2 & 0  & \tilde{f}_2 \\
% 0 & 0 & 1 & c_3/\beta_3 & \tilde{f}_3 \\
% 0 & 0 & 0 & 1 & \tilde{f}_4
% \end{array}
% \right] \sim
% \left[
% \begin{array}{cccc|c}
% 1 & c_1/\beta_1 & 0 & 0 & \tilde{f}_1 \\
% 0 & 1 & c_2/\beta_2 & 0  & \tilde{f}_2 \\
% 0 & 0 & 1 & 0 & \tilde{f}_3 -\frac{c_3}{\beta_3}\tilde{f}_4 \\
% 0 & 0 & 0 & 1 & \tilde{f}_4
% \end{array}
% \right]
% \end{equation}

%%% Listing
% \lstinputlisting[language=c++, firstline=146,
% lastline=158]{../problems.cpp}

%%% LU matrix, with diag dots for general case
% \begin{equation}
% A = LU =
% \begin{bmatrix}
% 1 & 0 & 0 & \dots & 0 & 0 \\
% l_{21} & 1 & 0 & \dots & 0 & 0 \\
% l_{31} & l_{32} & 1 & \dots & 0 & 0 \\
%   &\vdots & & \ddots & \vdots  & \\
% l_{n-11} & l_{n-12} & l_{n-13} & \dots & 1 & 0 \\
% l_{n1} & l_{n2} & l_{n3} & \dots & l_{nn-1} & 1
% \end{bmatrix}
% \begin{bmatrix}
% u_{11} & u_{12} & u_{13} & \dots & u_{1n-1} & u_{1n} \\
% 0 & u_{22} & u_{23} & \dots & u_{2n-1} & u_{2n} \\
% 0 & 0 & u_{33} & \dots & u_{3n-1} & u_{3n} \\
%   &\vdots & & \ddots & \vdots  & \\
% 0 & 0 & 0 & \dots & u_{n-1n-1} & u_{n-1n} \\
% 0 & 0 & 0 & \dots & 0 & u_{nn}
% \end{bmatrix}
% \end{equation}

%%% Table
% \begin{table}[h]
% \caption{Elapsed time for increasing $n$}
% \begin{tabular}{lcc}
% \hline
% n & TDMA [s] & LU [s] \\ \hline
% 10 & 0.0000035 & 0.00106083 \\
% 100 & 0.0000116 & 0.0022319 \\
% 1000 & 0.000077892 & 0.0677764 \\
% 10000 & 0.000878769 & 21.9247 \\
% 100000 & 0.00757418 & n/a \\
% 1000000 & 0.08616075 & n/a \\
% 10000000 & 0.76534 & n/a
% \end{tabular}
% \label{tab:solver_times}
% \end{table}

%%% Figure
% \begin{figure}[h]
  % \centering
  % \includegraphics[width=0.9\linewidth]{figures/relerror.png}
  % \caption{Plot of maximum relative error as a function of step size}
  % \label{fig:relerror}
% \end{figure}

%%% Cition and bilbliography%%
% # \emph{Thomas Algorithm} \cite{thomasalgo} How to cite.
% \begin{thebibliography}{10}
  % \bibitem{thomasalgo}{Thomas, L.H. (1949), \emph{Elliptic
        % Problems in Linear Differential Equations over a
        % Network}. Watson Sci. Comput. Lab Report, Columbia University,
      % New York.}
      % \bibitem{morten}{Hjorth-Jensen, M. (2015). \emph{Computational
        % Physics - Lecture Notes 2015}. University of Oslo}
    % \bibitem{golub}{Golub, G.H., van Loan, C.F. (1996). \emph{Matrix
          % Computations} (3rd ed.), Baltimore: John Hopkins.}
% \end{thebibliography}

\end{document}
